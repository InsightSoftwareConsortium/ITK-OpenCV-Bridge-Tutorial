\section{ITK OpenCV Bridge}


\centeredlargetext{white}{black}{
ITK OpenCV Bridge
}

\begin{frame}
\frametitle{Introduction}
\begin{itemize}
\item ITK Module for working with other libraries
\item Moving frame and/or video data between OpenCV and ITK
\item Bring biomedical and computer vision folks together
\item \url{https://github.com/itkvideo/ITK}
\end{itemize}
\end{frame}

\begin{frame}
\frametitle{Funding Source}
\begin{itemize}
\item Contract from the National Library of Medicine (HHSN276201000579P)
\item Algorithms, Adapters \& Data Distribution Outreach 2010: Increasing the Impact of the Insight Toolkit (ITK)
\end{itemize}
\end{frame}

\begin{frame}
\frametitle{Design Choices}
\begin{itemize}
\item OpenCV users and ITK users should both be comfortable
\item Image to image utility functions
\item cv::Source to itk::VideoStream
\item (Later) Focus on performance
\end{itemize}
\end{frame}

\centeredlargetext{white}{black}{
Basic Image Filtering (Revisited)
}


\begin{frame}
\frametitle{Include Header Files}
\begin{center}
We include ITK and OpenCV headers (like before):
\lstinputlisting[linerange={21-21}]{BasicFilteringOpenCVBridge.cxx}
\lstinputlisting[linerange={23-24}]{BasicFilteringOpenCVBridge.cxx}
\vspace{1 em}
We also need to include the bridge header:
\lstinputlisting[linerange={25-25}]{BasicFilteringOpenCVBridge.cxx}
\end{center}
\end{frame}

\begin{frame}
\frametitle{Basic Layout}
\begin{center}
The basic layout of this file is the same as the OpenCV Examples:
\lstinputlisting[linerange={28-36}]{BasicFilteringOpenCVBridge.cxx}
\lstinputlisting[linerange={62-75}]{BasicFilteringOpenCVBridge.cxx}
\end{center}
\end{frame}

\begin{frame}
\frametitle{Adding ITK}
\begin{center}
The type definitions should also be familiar from the ITK Material:
\lstinputlisting[linerange={38-45}]{BasicFilteringOpenCVBridge.cxx}
However, notice the bridge class. It contains the conversion function
between OpenCV and ITK.
\lstinputlisting[linerange={43-43}]{BasicFilteringOpenCVBridge.cxx}
\end{center}
\end{frame}

\begin{frame}
\frametitle{From OpenCV to ITK}
\begin{center}
We call our conversion function to go from a cv::Mat to an itk::Image
\lstinputlisting[linerange={47-48}]{BasicFilteringOpenCVBridge.cxx}
\end{center}
\end{frame}

\begin{frame}
\frametitle{Filtering with ITK}
\begin{center}
The median filtering is normal ITK code, but we do not connect our
output to a writer
\lstinputlisting[linerange={50-57}]{BasicFilteringOpenCVBridge.cxx}
\pause
Instead, we set it to our conversion function
\lstinputlisting[linerange={59-60}]{BasicFilteringOpenCVBridge.cxx}
\end{center}
\end{frame}
