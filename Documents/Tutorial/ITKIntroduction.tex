\section{ITK Introduction}


\centeredlargetext{white}{black}{
ITK Introduction
}

\begin{frame}
\frametitle{ITK is a Templated Library}
You will typically do:
\begin{itemize}
\item Include headers
\pause
\item Pick pixel type
\pause
\item Pick image dimension
\pause
\item Instantiate image type
\pause
\item Instantiate filter type
\pause
\item Create filters
\pause
\item Connect pipeline
\pause
\item Run pipeline
\end{itemize}
\end{frame}

{
\setbeamertemplate{navigation symbols}{}
\begin{frame}[fragile]
\frametitle{Basic Filtering}
\framesubtitle{Median Filter}
\begin{itemize}
\item Include header\footnote{Source code in Exercises/ITKIntroduction/exercise1/BasicImageFilteringITK.cxx}
\end{itemize}
\begin{center}
\lstinputlisting[linerange={19-21}]{BasicImageFilteringITK.cxx}
\end{center}
\pause
\begin{itemize}
\item Read images from files
\item Write images from files
\item Apply a median filter in an image
\end{itemize}
\end{frame}
}

\begin{frame}
\frametitle{How to Configure and Build}
\framesubtitle{cmake-gui}
\begin{itemize}
\item Create shortcut to build from nautilus
\item Shell script that launches gnome-terminal, cd s to binary dir and user can type "make"
\end{itemize}
\end{frame}


\begin{frame}
\frametitle{Basic Filtering}
\framesubtitle{Replace with Canny Filter}
\begin{center}
\lstinputlisting[linerange={21-23}]{BasicImageFilteringITK.cxx}
\end{center}
\end{frame}
