\section{OpenCV Introduction}


\centeredlargetext{white}{black}{
OpenCV Introduction
}


\centeredlargetext{white}{black}{
A Basic Image Filtering Example
}


\begin{frame}
\frametitle{Include Header Files}
\framesubtitle{OpenCVIntroduction/exercise1/BasicFilteringOpenCV.cxx}
\begin{center}
We need to include the OpenCV headers
\lstinputlisting[linerange={19-20}]{BasicFilteringOpenCV.cxx}
\pause
\vspace{1 em}
and standard library headers for {\tt cout} and {\tt string}
\lstinputlisting[linerange={22-23}]{BasicFilteringOpenCV.cxx}
\end{center}
\end{frame}


\begin{frame}
\frametitle{The Main function}
\framesubtitle{OpenCVIntroduction/exercise1/BasicFilteringOpenCV.cxx}
\begin{center}
%\lstinputlisting[linerange={26-34,42-45,54-57,60-63}]{BasicFilteringOpenCV.cxx}
\begin{itemize}
\lstinputlisting[linerange={26-51},firstnumber=26,numbers=left]{BasicFilteringOpenCV.cxx}
\end{itemize}
\end{center}
\end{frame}


\begin{frame}
\frametitle{Loading an Image / Applying a Filter}
\framesubtitle{OpenCVIntroduction/exercise1/BasicFilteringOpenCV.cxx}
\begin{center}
\begin{itemize}
\item If no arguments, print usage and exit \\
Note: {\tt argv[0]} contains the executable name
\lstinputlisting[linerange={28-32},firstnumber=28,numbers=left]{BasicFilteringOpenCV.cxx}
\pause
\item Load an image from the specified file into a matrix object
\lstinputlisting[linerange={34-34},firstnumber=34,numbers=left]{BasicFilteringOpenCV.cxx}
\pause
\item Create a matrix for the output and apply a median filter
\lstinputlisting[linerange={35-36},firstnumber=35,numbers=left]{BasicFilteringOpenCV.cxx}
\item The last argument represent the size of the filter, $9\times9$ in this case
\end{itemize}
\end{center}
\end{frame}


\begin{frame}
\frametitle{Displaying an Image}
\framesubtitle{OpenCVIntroduction/exercise1/BasicFilteringOpenCV.cxx}
\begin{center}
\begin{itemize}
\item If no output file is specified, use HighGUI to display the resulting image.
\lstinputlisting[linerange={38-44},firstnumber=38,numbers=left]{BasicFilteringOpenCV.cxx}
\pause
\item A title string is used to identify the GUI window.
\item Create a named window, CV\_WINDOW\_FREERATIO is needed for display with OpenGL.
\item Show the image in the previously created window.
\item Wait for the user to press a key, then continue.
\end{itemize}
\end{center}
\end{frame}


\begin{frame}
\frametitle{Saving an Image}
\framesubtitle{OpenCVIntroduction/exercise1/BasicFilteringOpenCV.cxx}
\begin{center}
\begin{itemize}
\item If an output file is specified, write the image to the specified file
\lstinputlisting[linerange={45-48},firstnumber=45,numbers=left]{BasicFilteringOpenCV.cxx}
\end{itemize}
\end{center}
\end{frame}


\begin{frame}[fragile]
\frametitle{Exercise 1}
\framesubtitle{OpenCVIntroduction/exercise1/BasicFilteringOpenCV.cxx}
\begin{center}
\begin{itemize}
\item Replace the median filter with a Canny edge detector
\lstinputlisting[linerange={35-36},firstnumber=35,numbers=left]{BasicFilteringOpenCV.cxx}
\pause
\item Hint: The OpenCV Canny function has this signature
\begin{lstlisting}
void Canny(const Mat& image, Mat& edges, double threshold1, double threshold2);
\end{lstlisting}
\pause
\item Canny requires grayscale image input, but our images are color
\item Hint: Use this function to convert a color space
\begin{lstlisting}
void cvtColor(const Mat& src, Mat& dst, int code);
\end{lstlisting}
\item Use {\tt CV\_BGR2GRAY} for {\tt code} to convert BGR color to gray.
\end{itemize}
\end{center}
\end{frame}

\begin{frame}[fragile]
\frametitle{Exercise 1: Answer}
\framesubtitle{OpenCVIntroduction/exercise1/BasicFilteringOpenCVAnswer.cxx}
\begin{center}
\begin{itemize}
\lstinputlisting[linerange={34-38},firstnumber=34,numbers=left]{BasicFilteringOpenCVAnswer.cxx}
\end{itemize}
\end{center}
\end{frame}
