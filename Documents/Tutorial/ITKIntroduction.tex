\section{ITK Introduction}


\centeredlargetext{white}{black}{
ITK Introduction
}

\begin{frame}
\frametitle{ITK is a Templated Library}
You will typically do:
\begin{itemize}
\item Include headers
\pause
\item Pick pixel type
\pause
\item Pick image dimension
\pause
\item Instantiate image type
\pause
\item Instantiate filter type
\pause
\item Create filters
\pause
\item Connect pipeline
\pause
\item Run pipeline
\end{itemize}
\end{frame}

{
\setbeamertemplate{navigation symbols}{}
\begin{frame}[fragile]
\frametitle{Basic Filtering - Median Filter}
\framesubtitle{ITKIntroduction/exercise1/BasicImageFilteringITK.cxx}
\begin{itemize}
\item Include headers
\end{itemize}
\lstlistingwithnumber{19}{21}{BasicImageFilteringITK.cxx}
\pause
\begin{itemize}
\item Read images from files
\item Write images from files
\item Apply a Median filter in an image
\end{itemize}
\end{frame}
}

{
\setbeamertemplate{navigation symbols}{}
\begin{frame}[fragile]
\frametitle{Basic Filtering - Median Filter}
\framesubtitle{ITKIntroduction/exercise1/BasicImageFilteringITK.cxx}
\begin{itemize}
\item Declare pixel types and image dimension
\lstlistingwithnumber{32}{35}{BasicImageFilteringITK.cxx}
\end{itemize}
\pause
\begin{itemize}
\item Declare input and output image types
\lstlistingwithnumber{37}{38}{BasicImageFilteringITK.cxx}
\end{itemize}
\end{frame}
}

{
\setbeamertemplate{navigation symbols}{}
\begin{frame}[fragile]
\frametitle{Basic Filtering - Median Filter}
\framesubtitle{ITKIntroduction/exercise1/BasicImageFilteringITK.cxx}
\begin{itemize}
\item Declare the types for reader and writer
\lstlistingwithnumber{40}{41}{BasicImageFilteringITK.cxx}
\end{itemize}
\pause
\begin{itemize}
\item Instantiate the reader and writer objects (source and sink)
\lstlistingwithnumber{43}{44}{BasicImageFilteringITK.cxx}
\end{itemize}
\pause
\begin{itemize}
\item Set input and output filenames
\lstlistingwithnumber{46}{47}{BasicImageFilteringITK.cxx}
\end{itemize}
\end{frame}
}

{
\setbeamertemplate{navigation symbols}{}
\begin{frame}[fragile]
\frametitle{Basic Filtering - Median Filter}
\framesubtitle{ITKIntroduction/exercise1/BasicImageFilteringITK.cxx}
\begin{itemize}
\item Declare the Median filter type
\lstlistingwithnumber{49}{50}{BasicImageFilteringITK.cxx}
\end{itemize}
\pause
\begin{itemize}
\item Create the filter
\lstlistingwithnumber{51}{51}{BasicImageFilteringITK.cxx}
\end{itemize}
\end{frame}
}

{
\setbeamertemplate{navigation symbols}{}
\begin{frame}[fragile]
\frametitle{Basic Filtering - Median Filter}
\framesubtitle{ITKIntroduction/exercise1/BasicImageFilteringITK.cxx}
\begin{itemize}
\item Define the Median kernel radius (Manhattan Radius)
\lstlistingwithnumber{55}{58}{BasicImageFilteringITK.cxx}
\end{itemize}
\pause
\begin{itemize}
\item Connect the pipeline
\lstlistingwithnumber{60}{61}{BasicImageFilteringITK.cxx}
\end{itemize}
\end{frame}
}

{
\setbeamertemplate{navigation symbols}{}
\begin{frame}[fragile]
\frametitle{Basic Filtering - Median Filter}
\framesubtitle{ITKIntroduction/exercise1/BasicImageFilteringITK.cxx}
\begin{itemize}
\item Trigger the pipeline execution by calling Update().
\lstlistingwithnumber{63}{71}{BasicImageFilteringITK.cxx}
\end{itemize}
\pause
\begin{itemize}
\item ITK uses C++ exceptions for error management
\item Exceptions are typically thrown during Update() calls
\item Applications must catch the exceptions and solve them
\end{itemize}
\end{frame}
}

\begin{frame}
\frametitle{How to Configure and Build}
\framesubtitle{cmake-gui}
\begin{itemize}
\item Create a binary directory
\item Configure the code with CMake
\item Build (compile and link an executable)
\item Run it in example image
\end{itemize}
\end{frame}

\begin{frame}[fragile]
\frametitle{How to Configure and Build}
\framesubtitle{cmake-gui}
\begin{itemize}
\item Create a binary directory
\begin{verbatim}
cd ~/bin
mkdir itkexercise1b
cd itkexercise1b
\end{verbatim}
\end{itemize}
\vspace{1in}
\textcolor{gray}{For convenience, a pre-built version is available in \texttt{\textasciitilde/bin/itkexercise1}}
\end{frame}

\begin{frame}[fragile]
\frametitle{How to Configure and Build}
\framesubtitle{cmake-gui}
\begin{itemize}
\item Run ``cmake-gui''
\end{itemize}
\begin{center}
  \includegraphics[width=0.5\paperwidth]{Screenshot-CMakeGUI-01.png}
\end{center}
\end{frame}

\begin{frame}[fragile]
\frametitle{How to Configure and Build}
\framesubtitle{cmake-gui}
\begin{itemize}
\item Set ``Source Directory'' (where the source code is)
\item Set ``Binary Directory'' (where to build the executable)
\item Click on ``Configure''
\end{itemize}
\begin{center}
  \includegraphics[width=0.7\paperwidth]{Screenshot-CMakeGUI-02.png}
\end{center}
\end{frame}

\begin{frame}[fragile]
\frametitle{How to Configure and Build}
\framesubtitle{cmake-gui}
\begin{itemize}
\item You will get an error message
\end{itemize}
\begin{center}
  \includegraphics[width=0.4\paperwidth]{Screenshot-CMakeGUI-03.png}
\end{center}
\begin{itemize}
\item Because the project needs ITK and OpenCV
\item and we have not provided ITK\_DIR or OpenCV\_DIR yet
\end{itemize}
\end{frame}


\begin{frame}[fragile]
\frametitle{How to Configure and Build}
\framesubtitle{cmake-gui}
\begin{itemize}
\item Provide the path to ITK in the ITK\_DIR variable
\item /home/tutorial/bin/ITKVideo/Release
\pause
\item Provide the path to OpenCV in the OpenCV\_DIR variable
\item /home/tutorial/bin/opencv/Release
\end{itemize}
\begin{center}
  \includegraphics[width=0.7\paperwidth]{Screenshot-CMakeGUI-04.png}
\end{center}
\end{frame}

\begin{frame}[fragile]
\frametitle{How to Configure and Build}
\framesubtitle{cmake-gui}
\begin{itemize}
\item Click on ``Configure''
\item Click on ``Generate''
\end{itemize}
\begin{center}
  \includegraphics[width=0.7\paperwidth]{Screenshot-CMakeGUI-05.png}
\end{center}
\end{frame}

\begin{frame}[fragile]
\frametitle{How to Build}
\framesubtitle{make}
\begin{itemize}
\item In the command line do:
\begin{verbatim}
cd  /home/tutorial/bin/itkexercise1b
make
\end{verbatim}
\end{itemize}
\end{frame}

\begin{frame}[fragile]
\frametitle{How to Run}
\framesubtitle{/home/tutorial/bin/itkexercise1b}
\begin{itemize}
\item While in the binary directory:
\begin{verbatim}
/home/tutorial/bin/itkexercise1b
\end{verbatim}
\item In the command line type:
\begin{verbatim}
./BasicImageFilteringITK          \
      ~/data/mandrillgray.png       \
      ./mandrillgrayMedian.png      \
      3  3
\end{verbatim}
\end{itemize}
\end{frame}

\begin{frame}[fragile]
\frametitle{How to View the Result}
\framesubtitle{Image viewing application ``eye of gnome'': eog}
\begin{itemize}
\item In the command line type:
\begin{verbatim}
eog ~/data/mandrillgray.jpg &
eog ./mandrillgrayMedian.png &
\end{verbatim}
\end{itemize}
\end{frame}

\begin{frame}[fragile]
\frametitle{Result of Median Filter}
\begin{center}
  \includegraphics[width=0.45\paperwidth]{Screenshot-mandrillgray-01.png}
  \includegraphics[width=0.45\paperwidth]{Screenshot-mandrillgrayMedian-01.png}
\end{center}
\end{frame}

\begin{frame}[fragile]
\frametitle{Excercise 1}
\framesubtitle{Replace the filter with another one}
\begin{itemize}
\item Select a Filter from the Doxygen documentation\\
(e.g. MeanImageFilter)
\item Replace the MedianImageFilter with the selected filter
\item Recompile
\item Rerun
\end{itemize}
\end{frame}

\begin{frame}[fragile]
\frametitle{ITK Doxygen Documentation}
\begin{center}
  \includegraphics[scale=0.3]{Screenshot-ITKDoxygen-01.png}
\end{center}
\end{frame}

\begin{frame}
\frametitle{Excercise 1}
\framesubtitle{ITKIntroduction/exercise1/BasicImageFilteringITKAnswer1.cxx}
\begin{itemize}
\item First we replace the Header file:
\lstlistingwithnumber{21}{21}{BasicImageFilteringITKAnswer1.cxx}
\pause
\item Then we replace the Filter instantiation:
\lstlistingwithnumber{49}{49}{BasicImageFilteringITKAnswer1.cxx}
\end{itemize}
\end{frame}

\begin{frame}[fragile]
\frametitle{How to Build}
\framesubtitle{make}
\begin{itemize}
\item In the command line do:
\begin{verbatim}
cd  /home/tutorial/bin/itkexercise1b
make
\end{verbatim}
\end{itemize}
\end{frame}

\begin{frame}[fragile]
\frametitle{How to Run}
\framesubtitle{/home/tutorial/bin/itkexercise1b}
\begin{itemize}
\item While in the binary directory:
\begin{verbatim}
/home/tutorial/bin/itkexercise1b
\end{verbatim}
\item In the command line type:
\begin{verbatim}
./BasicImageFilteringITK        \
      ~/data/mandrillgray.png     \
      ./mandrillgrayMean.png      \
      3  3
\end{verbatim}
\end{itemize}
\end{frame}

\begin{frame}[fragile]
\frametitle{How to View the Result}
\framesubtitle{Image viewing application ``eye of gnome'': eog}
\begin{itemize}
\item In the command line type:
\begin{verbatim}
eog ~/data/mandrillgray.jpg &
eog ./mandrillgrayMean.png &
\end{verbatim}
\end{itemize}
\end{frame}

\begin{frame}[fragile]
\frametitle{Result of Mean Filter}
\begin{center}
  \includegraphics[width=0.45\paperwidth]{Screenshot-mandrillgray-01.png}
  \includegraphics[width=0.45\paperwidth]{Screenshot-mandrillgrayMean-01.png}
\end{center}
\end{frame}

\begin{frame}[fragile]
\frametitle{Find All Other Exercises}
\begin{itemize}
\item Go to the binary directory
\begin{verbatim}
cd ~/bin/ITK-OpenCV-Bridge-Tutorial/Exercises
\end{verbatim}
\end{itemize}
\end{frame}

\begin{frame}
\frametitle{Basic Filtering - Canny Filter}
\framesubtitle{ITKIntroduction/exercise1/BasicImageFilteringITKAnswer2.cxx}
\begin{itemize}
\item Some filters expect specific pixel types
\pause
\item Canny Edge detection is an example
\pause
\item Here we Cast the image before Canny
\pause
\item Then we Cast/Rescale it after Canny
\end{itemize}
\end{frame}

\begin{frame}
\frametitle{Basic Filtering - Canny Filter}
\framesubtitle{ITKIntroduction/exercise1/BasicImageFilteringITKAnswer2.cxx}
\begin{itemize}
\item Let's start with the relevant headers:
\pause
\lstlistingwithnumber{21}{23}{BasicImageFilteringITKAnswer2.cxx}
\pause
\item We then declare the relevant pixel types
\pause
\lstlistingwithnumber{34}{36}{BasicImageFilteringITKAnswer2.cxx}
\pause
\item Then we declare the relevant image types
\pause
\lstlistingwithnumber{38}{40}{BasicImageFilteringITKAnswer2.cxx}
\end{itemize}
\end{frame}

\begin{frame}
\frametitle{Basic Filtering - Canny Filter}
\framesubtitle{ITKIntroduction/exercise1/BasicImageFilteringITKAnswer2.cxx}
\begin{itemize}
\item We declare the Casting filter and instantiate it:
\pause
\lstlistingwithnumber{52}{55}{BasicImageFilteringITKAnswer2.cxx}
\pause
\item We declare and instantiate the Canny filter:
\pause
\lstlistingwithnumber{58}{61}{BasicImageFilteringITKAnswer2.cxx}
\pause
\item and do the same for the RescaleIntensity filter:
\pause
\lstlistingwithnumber{64}{67}{BasicImageFilteringITKAnswer2.cxx}
\end{itemize}
\end{frame}

\begin{frame}
\frametitle{Basic Filtering - Canny Filter}
\framesubtitle{ITKIntroduction/exercise1/BasicImageFilteringITKAnswer2.cxx}
\begin{itemize}
\item We connect the pipeline:
\pause
\lstlistingwithnumber{70}{73}{BasicImageFilteringITKAnswer2.cxx}
\pause
\item Set the parameters of the Canny Edge detection filter:
\pause
\lstlistingwithnumber{76}{78}{BasicImageFilteringITKAnswer2.cxx}
\end{itemize}
\end{frame}

\begin{frame}
\frametitle{Basic Filtering - Canny Filter}
\framesubtitle{ITKIntroduction/exercise1/BasicImageFilteringITKAnswer2.cxx}
\begin{itemize}
\item Trigger the execution of the pipeline:
\pause
\lstlistingwithnumber{81}{89}{BasicImageFilteringITKAnswer2.cxx}
\pause
\item Note that the pipeline only runs when we call Update()
\pause
\item That's the point where we should catch exceptions
\end{itemize}
\end{frame}

\begin{frame}[fragile]
\frametitle{How to Run}
\framesubtitle{/home/tutorial/bin/itkexercise1b}
\begin{itemize}
\item While in the binary directory:
\begin{verbatim}
/home/tutorial/bin/itkexercise1b
\end{verbatim}
\pause
\item In the command line type:
\begin{verbatim}
./BasicImageFilteringITKAnswer2    \
      ~/data/mandrillgray.png        \
      ./mandrillgrayCanny.png        \
      6  1  8
\end{verbatim}
\pause
\item 6 = Variance for Gaussian
\item 1 = Lower Threshold
\item 8 = Upper Threshold
\end{itemize}
\end{frame}

\begin{frame}[fragile]
\frametitle{How to View the Result}
\framesubtitle{Image viewing application ``eye of gnome'': eog}
\begin{itemize}
\item In the command line type:
\begin{verbatim}
eog ~/data/mandrillgray.jpg &
eog ./mandrillgrayCanny.png &
\end{verbatim}
\end{itemize}
\end{frame}

\begin{frame}[fragile]
\frametitle{Result of Canny Filter}
\begin{center}
  \includegraphics[width=0.45\paperwidth]{Screenshot-mandrillgray-01.png}
  \includegraphics[width=0.45\paperwidth]{Screenshot-mandrillgrayCanny-01.png}
\end{center}
\end{frame}

