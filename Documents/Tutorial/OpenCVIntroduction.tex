\section{OpenCV Introduction}


\centeredlargetext{white}{black}{
OpenCV Introduction
}


\centeredlargetext{white}{black}{
A Basic Image Filtering Example
}


\begin{frame}
\frametitle{Include Header Files}
\framesubtitle{OpenCVIntroduction/exercise1/BasicFilteringOpenCV.cxx}
\begin{center}
We need to include the OpenCV headers
\lstinputlisting[linerange={19-20}]{BasicFilteringOpenCV.cxx}
\pause
\vspace{1 em}
and standard library headers for {\tt cout} and {\tt string}
\lstinputlisting[linerange={22-23}]{BasicFilteringOpenCV.cxx}
\end{center}

\end{frame}


\begin{frame}
\frametitle{The Main function}
\framesubtitle{OpenCVIntroduction/exercise1/BasicFilteringOpenCV.cxx}
\begin{center}
\lstinputlisting[linerange={26-34,42-45,54-57,60-63}]{BasicFilteringOpenCV.cxx}
\begin{itemize}
\item If no arguments, print usage and exit
\item If one argument, process an image and display the results
\item If two arguments, process an image and save to a file
\end{itemize}
\end{center}

\end{frame}


\begin{frame}
\frametitle{Loading an Image / Applying a Filter}
\framesubtitle{OpenCVIntroduction/exercise1/BasicFilteringOpenCV.cxx}
\begin{center}
\begin{itemize}
\item Load an image from the specified file into a matrix object
\end{itemize}
\lstinputlisting[linerange={36-37}]{BasicFilteringOpenCV.cxx}
\pause
\begin{itemize}
\item Create a matrix for the output and apply a median filter
\end{itemize}
\lstinputlisting[linerange={38-41}]{BasicFilteringOpenCV.cxx}
\end{center}

\end{frame}


\begin{frame}
\frametitle{Displaying an Image}
\framesubtitle{OpenCVIntroduction/exercise1/BasicFilteringOpenCV.cxx}
\begin{center}
\begin{itemize}
\item Use OpenCV HighGUI to display the resulting image
\end{itemize}
\lstinputlisting[linerange={43-54}]{BasicFilteringOpenCV.cxx}
\end{center}
\end{frame}


\begin{frame}
\frametitle{Saving an Image}
\framesubtitle{OpenCVIntroduction/exercise1/BasicFilteringOpenCV.cxx}
\begin{center}
\begin{itemize}
\item Write the image to the specified file
\end{itemize}
\lstinputlisting[linerange={55-60}]{BasicFilteringOpenCV.cxx}
\end{center}
\end{frame}


\begin{frame}[fragile]
\frametitle{Exercise 1}
\framesubtitle{OpenCVIntroduction/exercise1/BasicFilteringOpenCV.cxx}
\begin{center}
\begin{itemize}
\item Replace the median filter with a Canny edge detector
\end{itemize}
\lstinputlisting[numbers=left,linerange={38-41}]{BasicFilteringOpenCV.cxx}
\begin{itemize}
\item Hint: The OpenCV Canny function has this signature
\end{itemize}
\begin{lstlisting}
void Canny(const Mat& image, Mat& edges, double threshold1, double threshold2);
\end{lstlisting}
\begin{itemize}
\item Canny requires grayscale image input, but our images are color
\item Hint: Use this function to convert a color space
\end{itemize}
\begin{lstlisting}
void cvtColor(const Mat& src, Mat& dst, int code);
\end{lstlisting}
\end{center}
\end{frame}
