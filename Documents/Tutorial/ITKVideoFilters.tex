\section{ITK Video Filters}


\centeredlargetext{white}{black}{
ITK Video Filters
}

%%%%
% Introduction Page
%%%%
\begin{frame}
\frametitle{Introduction}
\begin{itemize}
\item Support video processing natively in ITK
\item Standard framework for multi-frame filters
\item Use ITK's library of image filters in video context
\end{itemize}
\end{frame}


%%%%
% Code walkthrough
%%%%
{
\setbeamertemplate{navigation symbols}{}
\begin{frame}[fragile]
\frametitle{Video Filtering - Median Filter}
\framesubtitle{ITKVideoPipeline/exercise1/Ex1\_ITKVideoPipeline.cxx}

\begin{itemize}
\item Video data structure
\lstlistingwithnumber{21}{21}{Ex1_ITKVideoPipeline.cxx}
\pause

\item Video file reader and writer
\item Use OpenCV for video IO
\lstlistingwithnumber{24}{26}{Ex1_ITKVideoPipeline.cxx}
\pause

\item Median image filter
\item Image filter $\rightarrow$ video filter wrapper
\lstlistingwithnumber{22}{23}{Ex1_ITKVideoPipeline.cxx}
\end{itemize}
\end{frame}
}

{
\setbeamertemplate{navigation symbols}{}
\begin{frame}[fragile]
\frametitle{Video Filtering - Median Filter}
\framesubtitle{ITKVideoPipeline/exercise1/Ex1\_ITKVideoPipeline.cxx}
\begin{itemize}
\item Types for data structures
\lstlistingwithnumber{36}{39}{Ex1_ITKVideoPipeline.cxx}
\pause

\item Types reader and writer
\lstlistingwithnumber{41}{42}{Ex1_ITKVideoPipeline.cxx}
\pause

\item Types for video median filter
\item Use image filter type to define video filter type
\lstlistingwithnumber{43}{45}{Ex1_ITKVideoPipeline.cxx}
\end{itemize}
\end{frame}
}

{
\setbeamertemplate{navigation symbols}{}
\begin{frame}[fragile]
\frametitle{Video Filtering - Median Filter}
\framesubtitle{ITKVideoPipeline/exercise1/Ex1\_ITKVideoPipeline.cxx}
\begin{itemize}
\item Create reader and writer
\lstlistingwithnumber{47}{48}{Ex1_ITKVideoPipeline.cxx}
\pause

\item Create image filter and video median filters
\lstlistingwithnumber{49}{50}{Ex1_ITKVideoPipeline.cxx}
\end{itemize}
\end{frame}
}

{
\setbeamertemplate{navigation symbols}{}
\begin{frame}[fragile]
\frametitle{Video Filtering - Median Filter}
\framesubtitle{ITKVideoPipeline/exercise1/Ex1\_ITKVideoPipeline.cxx}
\begin{itemize}
\item Set up reader and writer
\item Tell ITK that we're using OpenCV for IO
\lstlistingwithnumber{52}{54}{Ex1_ITKVideoPipeline.cxx}
\pause

\item Set up radius for (image) median filter
\item Set video filter wrapper to use image median filter internally
\lstlistingwithnumber{56}{60}{Ex1_ITKVideoPipeline.cxx}
\end{itemize}
\end{frame}
}

{
\setbeamertemplate{navigation symbols}{}
\begin{frame}[fragile]
\frametitle{Video Filtering - Median Filter}
\framesubtitle{ITKVideoPipeline/exercise1/Ex1\_ITKVideoPipeline.cxx}
\begin{itemize}
\item Connect the pipeline
\item reader $\rightarrow$ videoFilter $\rightarrow$ writer
\lstlistingwithnumber{62}{63}{Ex1_ITKVideoPipeline.cxx}
\pause

\item Try calling Update() to process the entire video
\lstlistingwithnumber{65}{73}{Ex1_ITKVideoPipeline.cxx}
\end{itemize}
\end{frame}
}


%%%%
% Excersize 1
%%%%
{
\setbeamertemplate{navigation symbols}{}
\begin{frame}[fragile]
\frametitle{Excercise 1}
\framesubtitle{Replace median filter with curvature flow filter}
\begin{itemize}
\item Hint 1: Curvature Flow requires {\tt float} as the pixel
  type for output.
\pause

\item Hint 2: OpenCV uses ffmpeg which requires {\tt unsigned char}
  as the pixel type for writing.
\pause

\item Hint 3: Curvature Flow does not take a radius parameter. It's
  salient functions are:
\lstlistingwithnumber{78}{79}{Ex1_ITKVideoPipelineAnswer.cxx}
\pause

\item Hint 4: To use an image filter in a video pipeline use
  {\tt ImageFilterToVideoFilterWrapper}
\end{itemize}
\end{frame}
}

{
\setbeamertemplate{navigation symbols}{}
\begin{frame}[fragile]
\frametitle{Excercise 1: Answer}
\framesubtitle{Replace median filter with curvature flow filter}
\begin{itemize}
\item Include curvature and cast image filters
\lstlistingwithnumber{22}{23}{Ex1_ITKVideoPipelineAnswer.cxx}
\pause

\item Re-define types using seperate IO and Real pixel types
\lstlistingwithnumber{36}{43}{Ex1_ITKVideoPipelineAnswer.cxx}
\end{itemize}
\end{frame}
}

{
\setbeamertemplate{navigation symbols}{}
\begin{frame}[fragile]
\frametitle{Excercise 1: Answer}
\framesubtitle{Replace median filter with curvature flow filter}
\begin{itemize}
\item Define cast filter to convert from Real to IO pixel type
\lstlistingwithnumber{47}{50}{Ex1_ITKVideoPipelineAnswer.cxx}
\pause

\item Replace median filter with curvature flow filter
\item Use Real pixel type for output
\lstlistingwithnumber{51}{54}{Ex1_ITKVideoPipelineAnswer.cxx}
\end{itemize}
\end{frame}
}

{
\setbeamertemplate{navigation symbols}{}
\begin{frame}[fragile]
\frametitle{Excercise 1: Answer}
\framesubtitle{Replace median filter with curvature flow filter}
\begin{itemize}
\item Set up video cast filter
\lstlistingwithnumber{68}{68}{Ex1_ITKVideoPipelineAnswer.cxx}
\pause

\item Set up curvature flow filter
\lstlistingwithnumber{70}{72}{Ex1_ITKVideoPipelineAnswer.cxx}
\pause

\item Connect the pipeline
\item reader $\rightarrow$ curvature
  flow $\rightarrow$ output caster $\rightarrow$ writer
\lstlistingwithnumber{74}{76}{Ex1_ITKVideoPipelineAnswer.cxx}
\end{itemize}
\end{frame}
}


%%%%
% Excersize 2
%%%%
{
\setbeamertemplate{navigation symbols}{}
\begin{frame}[fragile]
\frametitle{Excercise 2}
\framesubtitle{Compute frame differences after diffusion}
\begin{itemize}
\item Start with solution to Excersize 1 (Ex2\_ITKVideoPipeline.cxx)
\pause

\item Hint 1: Use FrameDifferenceVideoFilter
\pause

\item Hint 2: Difference filter's important parameter is:
\lstlistingwithnumber{73}{73}{Ex2_ITKVideoPipelineAnswer.cxx}
\item This sets the spacing of the frames to be differenced. Setting
  it to 1 means adjacent frames will be used.
\end{itemize}
\end{frame}
}

{
\setbeamertemplate{navigation symbols}{}
\begin{frame}[fragile]
\frametitle{Excercise 2: Answer}
\framesubtitle{Compute frame differences after diffusion}
\begin{itemize}
\item Include frame difference filter
\lstlistingwithnumber{25}{25}{Ex2_ITKVideoPipelineAnswer.cxx}
\pause

\item Define type for frame difference filter
\item Since this is a native video filter, no wrapper is necessary
\lstlistingwithnumber{57}{58}{Ex2_ITKVideoPipelineAnswer.cxx}
\pause

\item Create frame difference filter
\lstlistingwithnumber{66}{67}{Ex2_ITKVideoPipelineAnswer.cxx}
\end{itemize}
\end{frame}
}

{
\setbeamertemplate{navigation symbols}{}
\begin{frame}[fragile]
\frametitle{Excercise 2: Answer}
\framesubtitle{Compute frame differences after diffusion}
\begin{itemize}
\item Set the frame offset to 1
\lstlistingwithnumber{73}{73}{Ex2_ITKVideoPipelineAnswer.cxx}
\pause

\item Connect the pipeline
\item reader $\rightarrow$ curvature flow $\rightarrow$ output caster
  $\rightarrow$ frame differ $\rightarrow$ writer
\lstlistingwithnumber{81}{84}{Ex2_ITKVideoPipelineAnswer.cxx}
\end{itemize}
\end{frame}
}
